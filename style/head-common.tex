
%%%%%%%%%%%%%%%%%%%%%%%%%%%%%%%%%%%%%%
%%% SPACE SAVERS
%%%%%%%%%%%%%%%%%%%%%%%%%%%%%%%%%%%%%%
\usepackage[small,compact]{titlesec}
%%% left, before, after
%\titlespacing*{\section}{0pt}{3pt}{0pt}
%\titlespacing*{\subsection}{0pt}{2pt}{0pt}
\setlength{\textfloatsep}{0.08in}
\setlength{\floatsep}{0.06in}
%%% PATCH TO FIX NUMBERING (titlesec bug removes numbering)
% https://tex.stackexchange.com/questions/299969/titlesec-loss-of-section-numbering-with-the-new-update-2016-03-15
%\usepackage{etoolbox}
%\makeatletter
%\patchcmd{\ttlh@hang}{\parindent\z@}{\parindent\z@\leavevmode}{}{}
%\patchcmd{\ttlh@hang}{\noindent}{}{}{}
%\makeatother
%%% AMSMATH (align*) SPACING
%\expandafter\def\expandafter\normalsize\expandafter{%
%  \normalsize
%  \setlength\abovedisplayskip{1pt}
%  \setlength\belowdisplayskip{1pt}
%  \setlength\abovedisplayshortskip{1pt}
%  \setlength\belowdisplayshortskip{1pt}
%}

%%%%%%%%%%%%%%%%%%%%%%%%%%%%%%%%%%%%%%
%%% PACKAGES
%%%%%%%%%%%%%%%%%%%%%%%%%%%%%%%%%%%%%%
\usepackage{array}
\usepackage{fancyhdr}
\usepackage{booktabs}
\usepackage{amsmath}
\usepackage{amsthm}
\usepackage{amsfonts}
\usepackage{amssymb}
\usepackage{calrsfs}
\usepackage{pifont}
\usepackage{verbatim}
\usepackage{url}
\usepackage{hyperref}
\hypersetup{
    colorlinks=false,
    pdfborder={0 0 0},
    %pdftitle={Blocking-Resistant Network Services using \systemname},
    %pdfsubject={Network security and privacy; Peer-to-peer, overlay, and content distribution networks},
}
\usepackage{graphicx}
\usepackage{caption}
\usepackage{subcaption}
\usepackage{sidecap}
\usepackage{float}
\usepackage[section]{placeins}
\usepackage{enumitem}
\usepackage{color}
\usepackage{listings}
\usepackage{algorithm}
\usepackage{algpseudocode}
\usepackage{multirow}
\usepackage{xspace}

%%%%%%%%%%%%%%%%%%%%%%%%%%%%%%%%%%%%%%
%%% SETTINGS
%%%%%%%%%%%%%%%%%%%%%%%%%%%%%%%%%%%%%%
\lstset{
  %language=HTML,
  %basicstyle=\scriptsize\ttfamily,       % the size of the fonts that are used for the code
  basicstyle=\footnotesize\ttfamily,       % the size of the fonts that are used for the code
  %numbers=left,                   % where to put the line-numbers
  numbers=none,                   % where to put the line-numbers
  numberstyle=\scriptsize,        % the size of the fonts that are used for the line-numbers
  stepnumber=1,                   % the step between two line-numbers. If it is 1 each line will be numbered
  numbersep=5pt,                  % how far the line-numbers are from the code
  backgroundcolor=\color{white},  % choose the background color. You must add \usepackage{color}
  showspaces=false,               % show spaces adding particular underscores
  showstringspaces=false,         % underline spaces within strings
  showtabs=false,                 % show tabs within strings adding particular underscores
  frame=none,                     % adds a frame around the code
  tabsize=2,                      % sets default tabsize to 2 spaces
  captionpos=b,                   % sets the caption-position to bottom
  breaklines=true,                % sets automatic line breaking
  breakatwhitespace=false,        % sets if automatic breaks should only happen at whitespace
  escapeinside={\%*}{*},         % if you want to add a comment within your code
  commentstyle=\color{green},
  keywordstyle=\color{blue}\bfseries,
  stringstyle=\color{red}}

\lstdefinelanguage{javascript}{
  keywords={typeof, new, true, false, catch, function, return, null, catch, switch, var, if, in, while, do, else, case, break},
  keywordstyle=\color{blue}\bfseries,
  ndkeywords={class, export, boolean, throw, implements, import, this},
  ndkeywordstyle=\color{black}\bfseries,
  identifierstyle=\color{black},
  sensitive=false,
  comment=[l]{//},
  morecomment=[s]{/*}{*/},
  commentstyle=\color{purple}\ttfamily,
  stringstyle=\color{red}\ttfamily,
  morestring=[b]',
  morestring=[b]''
}

%%%%%%%%%%%%%%%%%%%%%%%%%%%%%%%%%%%%%%
%%% CUSTOM COMMANDS
%%%%%%%%%%%%%%%%%%%%%%%%%%%%%%%%%%%%%%
\newcommand{\squishlist}{\begin{itemize}[itemsep=0pt,parsep=0pt,topsep=0pt,partopsep=0pt,leftmargin=1em,labelwidth=1em,labelsep=0.5em]}
\newcommand{\squishlistend}{\end{itemize}}
\newcommand{\squish}{\begin{itemize}[itemsep=0pt,parsep=0pt,topsep=0pt,partopsep=0pt,leftmargin=1em,labelwidth=1em,labelsep=0.5em]}
\newcommand{\squishend}{\end{itemize}}
\newcommand{\squishenum}{\begin{enumerate}[itemsep=0.5pt,parsep=0pt,topsep=0pt,partopsep=0pt,leftmargin=1.5em,labelwidth=1em,labelsep=0.5em]{}}
\newcommand{\squishenumend}{\end{enumerate}}

\newcommand{\todo}[1]{[[[{\bf{TODO:\@#1}}]]]}
\newcommand\myurl[2]{\url{#1}}

\renewcommand{\floatpagefraction}{0.95}

\newcommand{\captionfonts}{\small}
\makeatletter  % Allow the use of @ in command names
\long\def\@makecaption#1#2{%
  \vskip 0.1in
  \sbox\@tempboxa{{\captionfonts{} #1: #2}}%
  \ifdim{} \wd\@tempboxa{} >\hsize
    {\captionfonts{} #1: #2\par}
  \else
    \hbox{} to\hsize{\hfil\box\@tempboxa\hfil}%
  \fi
  \vskip 0in}
\makeatother   % Cancel the effect of \makeatletter

% Theorems
\newtheorem{mydefinition}{Definition}
\newtheorem{mytheorem}{Theorem}
